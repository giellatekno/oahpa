
\documentclass{beamer}
\usepackage{ucs}
\usepackage[utf8x]{inputenc}
\usepackage[T1]{fontenc}

\usepackage{graphicx}
\usepackage{tipa}

\begin{document}
\title{Constraint Grammar in Dialogue Systems}   
\author{Lene Antonsen, Saara Huhmarniemi, Trond Trosterud \\
Centre for Sámi Language Technology \begin{figure}  \scalebox{0.10}[0.10]{\includegraphics{img/LogoEngelsk}} \end{figure} 
}
\date


\frame{\titlepage} 

\frame{\frametitle{Contents}\tableofcontents} 


\section{1. Introduction} 

\frame{\frametitle{Parser-based CALL programs}
Parser-based CALL programs for learners of North Sámi based on pre-existing LT resources developed at the University of Tromsø:
\begin{itemize}
\item finite state morphological analyser/generator (fst)
\item constraint grammar (CG) parser
\item number word generator (xfst)
\end{itemize} 

\vspace{0.5cm}
The morphological analyser/generator is implemented with fst and compiled with the Xerox compilers twolc and lexc. \\
The morphological disambiguator is implemented in the CG-framework. \\
The OAHPA! platform is implemented in Django, a Python-based web development framework, combined with a Mysql database.
}

\frame{\frametitle{North Sámi VISL} 
\scalebox{0.24}[0.24]{\includegraphics{img/oahpavisl.png}} \\
}


\frame{\frametitle{http://oahpa.uit.no/} 
\scalebox{0.30}[0.30]{\includegraphics{img/Oahpanew.png}} \\
}


\section{2. Analysing the user's input}

\frame{\frametitle{Pedagogical lexicon} 
\scalebox{0.40}[0.40]{\includegraphics{img/nounlexiconMonni.png}} \\ (egg)
}



\frame{\frametitle{Sentence generator} 
used in \textbf{Vasta} -- the question-answer drill and in \textbf{Morfa-C} -- morphological exercises in a sentential frame \\
\vspace{0.5cm}
\scalebox{0.50}[0.50]{\includegraphics{img/sentencegenerator.png}} \\
\vspace{0.5cm}
(MAINV question-particle SUBJ yesterday)
}

\frame{\frametitle{Sentence generator in Vasta} 
\scalebox{0.50}[0.50]{\includegraphics{img/Vasta_sentencegen_example.png}} \\
\vspace{0.5cm}
("Did the boy go to church yesterday?" \\
"No, he does not.")
}

\frame{\frametitle{Constraint Grammar parser -- vislcg3}
The analyser for Vasta and Sahka is based upon a pre-existing sámi CG paser compiled with vislcg3. \\
\vspace{0.5cm}
CG is robust enough for handling unconstrained input, and at the same time accurate enough to identify errors. \\
}


\frame{\frametitle{Sahka -- the dialogue program} 
\scalebox{0.39}[0.41]{\includegraphics{img/TVhivssegis.png}} \\ 
\vspace{0.5cm}
Question: "In which room should we place the TV?" \\
Answer: "We should place it in the toilet."
}


\frame{\frametitle{Schematical view of the process} 
\scalebox{0.65}[0.65]{\includegraphics{img/qa2.pdf}} \\
}


%\frame{\frametitle{Morphological analysis} 
%\scalebox{0.35}[0.35]{\includegraphics{img/GudelatnjiiQ.png}{img/GudelatnjiiB.png}}\scalebox{0.35}[0.35]{\includegraphics{img/GudelatnjiiA.png}{img/GudelatnjiiB.png}} \\
%}

\section{3. Navigation and collecting information}

\frame{\frametitle{Assignment of navigation tag} 
\scalebox{0.37}[0.37]{\includegraphics{img/hivssegisCGanal.png}} \\
\scalebox{0.35}[0.35]{\includegraphics{img/hivsset_tag.png}} \\
}

\frame{\frametitle{Navigating in the dialogue -- alternative links} 
\scalebox{0.35}[0.35]{\includegraphics{img/hivsset_tag.png}} \\
\scalebox{0.55}[0.55]{\includegraphics{img/gosabidjatTV.png}} \\
Question: "In which room should we place the TV?" \\
Alt. WC: "That is not a good idea. Make a new try." \\ 
Default: "We carry it there together." 
}



\frame{\frametitle{Target tag} 
What do you want to drink?\\
\vspace{0.5cm}
\scalebox{0.4}[0.4]{\includegraphics{img/target_acc.png}} \\
}


\frame{\frametitle{Tags for affirmative and negative answer} 
Do you have children? \\ What do you want to drink?\\ 
\vspace{0.5cm}
\scalebox{0.5}[0.5]{\includegraphics{img/aff_or_neg_colours.png}} \\
}

\frame{\frametitle{Tag for omitting the next question} 
Do you have children? How many children do you have?\\
Do you work? What kind of work do you have?\\
Do you have a car? What kind of car do you have?\\
\vspace{0.5cm}
\scalebox{0.55}[0.55]{\includegraphics{img/pass_rules.png}} \\
}


\frame{\frametitle{Collecting information from the input} 
What is your name?\\ What car do you have? \\ Where do you live? \\
\vspace{0.5cm}
\scalebox{0.45}[0.45]{\includegraphics{img/picking_name_new.png}} \\
}

\frame{\frametitle{Using the information} 
What car do you have? \\
\scalebox{0.5}[0.5]{\includegraphics{img/what_car.png}} \\
}

\frame{\frametitle{Navigating in the dialogue -- alternative branches} 
\scalebox{0.45}[0.45]{\includegraphics{img/pickingage_colours.png}} \\
\vspace{0.5cm} 
\scalebox{0.45}[0.45]{\includegraphics{img/age_branching.png}} \\
("How old are you?")
}

\section{4. Grammar feedback}


\frame{\frametitle{Disambiguation and assignment of grammar tag} 
\scalebox{0.3}[0.3]{\includegraphics{img/hivssegisCGanal.png}} \\
\scalebox{0.35}[0.35]{\includegraphics{img/missingIll.png}} \\
}

\section{5. Challenges}

%\frame{\frametitle{multi} 
%\scalebox{0.38}[0.38]{\includegraphics{img/missingIll.png}} \\
%}

\frame{\frametitle{Misspellings} 
\begin{itemize}
\item non-existing word
\item unintended word form
\item unintended lemma
\end{itemize}
}

\frame{\frametitle{Misspelling: non-existing word} 
\begin{itemize}
\item feedback: "X is not in our lexicon. Could it be a typo?"  
\item What is your name? $\rightarrow$ we accept anything and use the input string: "Good morning, X."  
\item Where do you live? \\
-- placename has not big initial letter  $\rightarrow$ fst with placenames with small initial letter and Lowercase-error-tag: "Remember that placenames are written with capital letter."\\  
-- placename is misspelled or it is not in our lexicon   $\rightarrow$ navigation to question: "I haven't heard about X. Is it a place?" -- default: next question. 
\end{itemize}
}

\frame{\frametitle{Misspelling: non-existing word} 
\begin{itemize}
\item One answer to this challenge is to utilize the North Sámi speller
\item Problem: it gives too many suggestions
\item Possible solution: Constraint the suggestions by CG rules 
\item Problem: The speller is made for native speakers 
\item Possible solution: Make a speller for learners 
\end{itemize}
}


\frame{\frametitle{Misspellings: unintended word form} 
The challenge is to give a feedback according to what the user thinks she has written. \\
Some are systematic: 
\begin{itemize}
\item Sg Loc vs. Nom PxSg3 -- vieljas vs. vielljas (viellja = brother) $\rightarrow$ "Do you mean locative? Remember consonant gradation." 
\item Pl Nom vs. Nom PxSg2 -- vieljat vs. vielljat (viellja = brother) $\rightarrow$ "Do you mean pluralis? Remember consonant gradation." 
\item Prs Sg1 vs. PrfPrt -- boran Prs Sg1 vs. borran PrfPrt (borrat = to eat). $\rightarrow$ "Do you mean present tense? Remember consonant gradation."
\end{itemize}
}

\frame{\frametitle{Misspellings: unintended lemma} 
Some are systematic: 
\begin{itemize}
\item ConNeg -- in bora (borrat = to eat) vs. in borat (boradit = to have a meal) \\ $\rightarrow$ "You should answer with the same verb as in the question."  \\ $\rightarrow$ "The negative form is not correct."  
\item Illative with weak gradation gives V Prt Sg3 -- skuvlii (skuvla N = school) vs. skuvllii (skuvllet V = to work as a teacher)  
\item viesut N Pl Nom (viessu = house) vs. viessut Imprt Pl1 (viessat = become tired)
\end{itemize}
}

%For misspellings that produce another word form of the same lemma, we have written rules that are based on the grammatical context. The real problem emerges when the spelling error gives rise to an unintended lemma. Then the challenge is to give a feedback according to what the student thinks she has written. In this case, feedback has to be tailored using the knowledge about the student’s interlanguage. We have created sets for typical unintended lemmata. Combined with contextual rules we can then give the user a good feedback due to the misspelling instead of the unintended lemma.
%
%E.g. if the student uses the Sg2 form of the main verb after the negative verb, instead of the correct ConNeg form, then the erroneous form can be a ConNeg form of a derivated verb, and the normal feedback will be: "You should answer with the same verb as in the question." The student will not understand this, because she thinks that the word form in the answer is an instance of the same verb. The solution was to generate all these forms of the verbs in the questions, make a set of them, and make a rule for in the right context, give the feedback: "The negative form is not correct." 


\frame{\frametitle{Answer with the same verb} 
We want the user to answer with the same verb, but we have to make exeptions for: 
\begin{itemize}
\item pro-verbs -- "What did you do yesterday?" 
\item some auxiliary verbs
\end{itemize} 
\vspace{0.5cm} 
We use regular expression: \\
\scalebox{0.38}[0.40]{\includegraphics{img/verblemma.png}} \\

}



\frame{\frametitle{Orthographic variation of words} 
\begin{itemize}
%\item vuola : vuollaga vs. vuolla : vuola (beer), in word lists
\item dárbbašit vs. dárbbahit, rf. the same verb rule
\end{itemize}
}

%\frame{\frametitle{Answer in the correct person} 
%Vasta and Sahka are different. This is for Vasta: \\
%\vspace{0.5cm}
%QPN - question's person-number \\
%APN - answer's person-number\\
%\vspace{0.5cm}
%\begin{tabular}[t]{ll|ll|ll}
%QPN &APN &QPN &APN &QPN &APN \\
%\hline
%Sg1 &Sg2 &\textbf{Du1} &\textbf{Du2} &\textbf{Pl1} &\textbf{Pl2} \\
%Sg2 &Sg1 &Du2 &Du1 &Pl2 &Pl1 \\
%\hline
%Sg3 &Sg3 &Du3 &Du3 &Pl3 &Pl3 \\
%\hline
%\end{tabular}
%}

\frame{\frametitle{Meta comments} 
\begin{itemize}
\item "Answering \textit{I-don't-know} is too simple. Try again."
\item "Your answer must always contain a finite verb. Could there be a typo in the verbform?"
\item "You must use one of the words in the wordlist in the left margin."
\item "You have not used the correct adjective. Try again." 
\item The user can quit the dialogue in a proper way by using the verb "heaitit" (= to quit) -- then the system navigates to the closing utterance of the dialogue (to be implemented)
\end{itemize}
%\hspace{0.5cm} \scalebox{0.45}[0.45]{\includegraphics{img/diastop.png}} \\

}


\frame{\frametitle{The grammar errors we have rules for 1} 
Verbs and their arguments
\begin{itemize}
\item verbs: finite, infinite, negative form, correct person/tense according to the question
\item case of argument based upon the interrogative 
\item case of argument based upon valency
\item locative vs. illative based upon movement
\item subject/verbal agreement
\end{itemize}
}

\frame{\frametitle{The grammar errors we have rules for 2} 
Other
\begin{itemize}
\item agreement inside NP 
\item numeral expressions: case and number  
\item PP: case of noun, pp based upon the interrogative  
\item time expressions 
\item special adverbs  
\item particles according to word order 
\item comparision of adjectives
\end{itemize}
}



\section{6. Evaluation}
\frame{\frametitle{Evaluation 1}
The programs are free available at internet. Appr. 500 queries/day. \\
\begin{table}
\begin{tabular}{|c|c|c|c|c|c|}
\hline
Morfa-S & Leksa & Sahka & Numra & Morfa-C & Vasta \\
41\% & 27\% & 13\% & 12\% & 5\% & 2\% \\
\hline
\end{tabular}
\end{table}
}

\frame{\frametitle{Evaluation 2}
The system has identified an error in the user's input:
\begin{table}
\begin{tabular}{|l|c|c|c|}
\hline 
\textbf{Rule type}  & \textbf{correct} & \textbf{wrong}   & \textbf{corr. \% }  \\
\hline 
wrong tense         & 7     & 0     & 100,0     \\ 
wrong V after neg   & 3     & 0     & 100,0     \\ 
no infinite V       & 1     & 0     & 100,0     \\ 
\hline 
orth. error         & 44    & 2     & 95,7      \\
wrong case for V-arg  & 26    & 4     & 86,7      \\
no finite verb        & 19    & 4     &  82,6 \\
\hline 
wrong S-V agreement   & 17    & 8     & 68,0 \\
wrong V choice        & 7     & 4     & 63,6 \\
\hline 
wrong word            & 4     & 4     & 50,0 \\
wrong case after Num  & 1     & 1     & 50,0 \\
\hline
\end{tabular}
\end{table}
}

\frame{\frametitle{Evaluation 3}
Which rules are not in use? Why? \\
\begin{itemize}
\item agreement inside NP (except for numeral expressions)
\item PP: case of noun, pp based of the interrogative  
\item time expressions 
\item particles according to word order 
\end{itemize}

}


\frame{\frametitle{An alternative to free input: http://e-tutor.org/} 
\scalebox{0.4}[0.4]{\includegraphics{img/e-tutor.png}} \\
}

\frame{\frametitle{Evaluation 4: Precision and recall}
\begin{table}%[htdp]
%\caption{default}
%\begin{center}
\begin{tabular}{|l|r|r|r|r||r|r|r|r|r|}
\hline
Error type	& tp		& fp		& tn		& fn	& prec	 & rec.	& acc.	& F-ms. \\
\hline
Gramm. err.   &   641   &   234   &   769    &   7    &   0,73   &   0,99   &   0,85   &   0,84	  \\
Sem. err.     &   805   &   69    &   764    &   12   &   0,92   &   0,99   &   0,95   &   0,95		  \\
Orth. err     &   875   &   0     &   776    &   0    &   1      &   1      &   1      &   1					  \\
Other err.    &   695   &   180   &   751    &   25   &   0,79   &   0,97   &   0,88   &   0,87	  \\
\hline
  &   3016  &   483   &   3060   &   44   &   0,86   &   0,98   &   0,92   &   0,92			  \\
\hline
\end{tabular}
%\end{center}
%\label{default}
\end{table}%

The high recall compared to the somewhat lower precision indicates that the system is a bit too critical towards the students:
\begin{itemize}
\item{It almost never lets through a (targeted) mistake, with the price of flagging some correct answers as errors.}
\end{itemize}
}


\section{7. Future perspectives}
\frame{\frametitle{Future perspectives}
How to improve the system? \\
\begin{itemize}
\item speller for misspellings 
\item grammartasks a la \textit{e-tutor}?
\item Vasta: decide what words the user should use, ala e-tutor as a supplement
\item Vasta: the user can choice topic instead of grammar tasks
\item Make the programs for more sámi languages
\item Classroom studies
\end{itemize}

}




\section{8. Conclusion}
\frame{\frametitle{Conclusion}

\begin{itemize}
\item By using a sloppy version of the syntactical analyser for North Sámi, combined with a set of error-detection rules, we have been able to build a flexible CALL resource. \\ 
\item The precision is not good enough
\item We need some kind of speller or a sloppy fst with errortags
\item "Totally" free input not always the best
\end{itemize}
}

\frame{\frametitle{Thanks to} 
The faculty of Humanities at the University of Tromsø, and the Sámi Parliament in Norway, for funding the project. \\
Berit Ánne Bals Baal 
} 

\frame{\frametitle{Centre for Sámi Language Technology} 
http://giellatekno.uit.no/ \\
http://oahpa.uit.no/

} 


%\frame{\frametitle{References} 
%\scalebox{0.32}[0.32]{\includegraphics{img/ref1.png}}
%\scalebox{0.32}[0.32]{\includegraphics{img/ref2.png}}





\end{document}

