
\documentclass[landscape,norsk,11pt]{seminar} 
 
\def\everyslide{\sf}
\usepackage{babel}
\usepackage{ucs}
\usepackage[utf8x]{inputenc}

\usepackage[T1]{fontenc}

\usepackage{hyperref}
\usepackage{graphics}

\slideframe{none}

\title{Interaktive digitale læremiddel for sørsamisk}

\author{Lene Antonsen, Biret Ánne Bals \\
Saara Huhmarniemi, Trond Trosterud \\
  \scalebox{0.25}[0.25]{\includegraphics{img/logoWeb070sh.jpg}} \\
  \textit{http://giellatekno.uit.no/}}

\begin{document}
\begin{slide}

\maketitle

\newslide
\textbf{Presenting our games}

Numra, Leksa, Morfa, \textbf{Vasta, Sahka}

\newslide
\textbf{Integrating the spellchecker in our ped program}


\newslide
\textbf{Unintended lemmas, lexical level}
Advice: How to cope with unintended lemmas?

Problem

e.g. $viessut$ (a rare verb)

\begin{itemize}
\item{Remove from the analyser, and get a question mark}
\item{Make a lexeme-specific rule for the $viessu-viessut$ pair}
\item{If lemma is from a set, say NOUN-PROBABLY-VERB, then change reading}
\end{itemize}


\newslide
\textbf{Marginal morphological analyses}

\begin{itemize}
\item{Px \& strong grade pro Loc and weak grade}
\end{itemize}


\newslide

QA information retrieval how to process questions
interactive, aske the user for clarification

QA: 

Question matrixes: generate a set of questions


\newslide
\textbf{The goal is to train morphology Therefore:}

\begin{itemize}
\item{No elipsis							  }
\item{Finite verb compulsatory}
\item{No inclusive 1st person plural}
\item{Didactics more important than pragmatics}
\item{The answer "I do not know" is not accepted}
\end{itemize}

\newslide
\textbf{Steps}

\begin{enumerate}
\item{Analyse (m-disambiguate) the questions and answers together}
\item{The disambiguation is incomplete, since we are careful with the errouneous input}
\item{Select the relevant reading}
\item{Make \&err assignment rules}
\item{Map \&err tag.}
\end{enumerate}

\newslide
\textbf{Answer with the same verb}

Solution: Sticky tag with regex (thanks to Tino)

Exceptional handling of pro-verbs 

\newslide
\textbf{Time expressions}


\newslide
\textbf{Names}


\newslide
\textbf{Age, regex}

\newslide

The errors are ordered:

First, we go for the finite verb


The user gets one error message at a time, corrects according to that, and eventually gets a new error message again.



Common CG up until mapping, but refraining from s-mapping
Instead, we map the \&err tags.



\newslide
\textbf{Sahka}

Default:

Yes, no, and one of them as default








\end{slide}
\end{document}




