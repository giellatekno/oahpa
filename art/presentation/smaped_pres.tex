
\documentclass[landscape,norsk,11pt]{seminar} 
 
\def\everyslide{\sf}
\usepackage{babel}
\usepackage{ucs}
\usepackage[utf8x]{inputenc}

\usepackage[T1]{fontenc}

\usepackage{hyperref}
\usepackage{graphics}

\slideframe{none}

\title{Interaktive digitale læremiddel for sørsamisk}

\author{Lene Antonsen, Saara Huhmarniemi, Trond Trosterud \\
  \scalebox{0.25}[0.25]{\includegraphics{img/logoWeb070sh.jpg}} \\
  \textit{http://giellatekno.uit.no/}}

\begin{document}
\begin{slide}

\maketitle



\newslide

\begin{table}[htdp]
\caption{Samisk språkteknologi ved UiT}
\begin{center}
\begin{tabular}{ll}
1998 &  Standardisert nordsamisk tastatur \\
     &  kildinsamiske bokstavar i Unicode \\
1999 &  Sørsamisk morfologisk analyseprogram \\
2001- &  Humfak: 1 fulltidsstilling, nord- og lulesamisk morfologi \\
2004-06 &  Humfak: 2,5 stilling, nord- og lulesamisk syntaks \\
2005-07 &  Sametinget: 4,5 stillingar, nord-, lulesamisk retteprogram \\
2006-08 &  Sametinget: 5 stillingar, også sørsamisk retteprogram \\
2008- &  UiT: 3 stillingar, samisk språkteknologisk senter \\
\end{tabular}
\end{center}
\label{default}
\end{table}%

\newslide

\textbf{Interaktive digitale læremiddel}

1 1/2 års prosjekt

Finansiert av Sametinget og UiT

Bakgrunn: Dårlege eksisterande hjelpemiddel

Mål: utnytte dei grammatiske analysatorane våre i undervisningssamanheng

(Verbbøying, negasjonsverb, øving)




\newslide

\textbf{Korleis utvide til sørsamisk}

\newslide

\textbf{Typar av digitale læremiddel}

\begin{enumerate}
\item{Vanlege papirbaserte læremiddel, distribuert via nettet (pdf)}
\item{Interaktive spel der datamaskina ikkje kan grammatikk}
\item{Interaktive spel der datamaskina kan grammatikk}
\end{enumerate}



\newslide
\textbf{Filosofien bak de pedagogiske programma for nordsamisk}

Ikkje sjølvstendige ''læreverk``, men verkty

Hjelpemiddel for differensiering


\newslide

\textbf{Mål får læringa}

\begin{enumerate}
\item{lære grammatikk}
\item{lære språk}
\end{enumerate}

\newslide

\textbf{Grammatikkspela våre}

\begin{enumerate}
\item{lære ordklasser}
\item{lære setningsanalyse}
\item{finst tilsvarande spel for norsk - mogleg å bruke både i norsk- og samisktimane}
\end{enumerate}

\url{http://visl.sdu.dk/}

\url{http://tekstlaboratoriet.uit.no/grei/}


\newslide

\textbf{Samarbeid med sørsamiske lærarar og undervisningsinstitusjonar?}

\begin{itemize}
\item{Oahpa-prosjektet for nordsamisk blir ferdig i 2008}
\item{Divvun II vil ha ein fungerande sørsamisk analysator om eit år}
\begin{itemize}
\item{Resultat: Infrastruktur for nordsamisk, analysatorar for sørsamisk}
\item{... men eit evt. sørsamisk prosjekt treng samarbeidspartnarar}
\end{itemize}
\end{itemize}


\newslide

\textbf{Frå sørsamisk perspektiv}

Godt høve til å utnytte ressursar som allereie finst til å få gode interaktive program

Mogleg å tilpasse til opplæringa som foregår i sørsamisk, både i grunnskolen, videregående skole og på kurs for voksne.

\newslide

\textbf{Innhald i samarbeidet}
\begin{itemize}
\item{Samarbeid om grunnlagsmateriale}
\item{Samarbeid om terminologi}
\item{Samarbeid om digitale program integrert i eksisterande eller nye læreverk}
\end{itemize}

\newslide

\textbf{Samarbeidsform}

(nettverk av) einskildlærarar

(nettverka av) sørsamiske institusjonar

\begin{itemize}
\item{lettare å koordinere på sørsamisk side}
\item{språklege og grammatiske spørsmål gjennomdiskutert på sørsamisk side}
\item{lettare å søke om pengar til samarbeidsprosjekt}
\end{itemize}



\newslide
\textbf{Heimesidene våre}

\url{http://giellatekno.uit.no/oahpa/}

\url{http://giellatekno/doc/ped/}


\end{slide}
\end{document}





