
\documentclass[landscape,norsk,11pt]{seminar} 
 
\def\everyslide{\sf}
\usepackage{babel}
\usepackage{ucs}
\usepackage[utf8x]{inputenc}

\usepackage[T1]{fontenc}

\usepackage{hyperref}
\usepackage{graphics}

\slideframe{none}

\title{Mo oažžut dihtora hálddašit lunddolaš giela válljenvejolašvuođaid giellaoahppanproseassas?  Gielalaš ja pedagogalaš čuolmmat.}

\author{Lene Antonsen, Biret Ánne Bals Baal\\
Saara Huhmarniemi, Trond Trosterud \\
 \scalebox{0.10}[0.10]{\includegraphics{img/LogoSamisk}}}
% \textit{http://giellatekno.uit.no/oahpa/}}
%  \scalebox{0.10}[0.10]{\includegraphics{img/vasta.png}} \\
\date{}
\begin{document}
\begin{slide}

\maketitle


\newslide
\textit{http://giellatekno.uit.no/oahpa/}
\scalebox{0.30}[0.30]{\includegraphics{img/gtoahpa.png}} 


\newslide
\textbf{VISL-prográmmat: oahppat grámmatihkka}\\
\newline
Sátneluohkáid, syntávssa\\


\newslide
\textbf{OAHPA-prográmmat: oahppat sámegiela}\\
\newline
\textbf{Leksa}: Sátnequiz - sámi/dáru ja dáru/sámi\\
\textbf{Numra}: Hárjehallat loguid\\
\textbf{Morfa}: Hárjehallat sojahit sániid, maid konteavsttas \\
\textbf{Vasta}: Hárjehallat vástidit jearaldagaide \\
\textbf{Sahka}: Hárjehallat ságastallat dihto fáttás

\newslide
\textit{http://victorio.uit.no/oahpa/morfa/}
\scalebox{0.35}[0.35]{\includegraphics{img/oahpa.png}} 

\newslide
\textbf{Vasta -- hárjehallat vástidit jearaldagaide}
\scalebox{0.90}[0.90]{\includegraphics{img/vasta.png}} \\

%\newslide
%\textbf{Generating questions}
%\scalebox{.35}[.35]{\includegraphics{img/xml_question.png}}

\newslide
\textbf{Pedagogalaš prográmmain ii leat dábálaččat giellateknologiija, muhto} 
\begin{itemize}
\item multiple choice 
\item stringmatch -- (\textit{viesus} = 6 mearkka) 
\end{itemize}

\textbf{Giellateknologiija:}
\begin{itemize}
\item analysa (\textit{viesus} = viessu N Sg Loc) 
\end{itemize}

\newslide
\textbf{Áigumuš:}\\
Prográmma galgá bagadit geavaheaddji seammá ládje go oahpaheaddji dahká.

\newslide
\textbf{Maid don lohket ikte?} \\
Dohkálaš vástádusat:
\begin{itemize}
\item Mun han lohken ollu \'aviissaid. 
\item Ikte mun gal lohken buori girjji. 
\item In lohkan maidege. 
\item Ikte in lohkan.
\end{itemize}

\newslide
\textbf{Maid don lohket ikte?}\\
Vasta-prográmma bagada go vástádus ii leat dohkálaš:
\begin{itemize}
\item Mun lohket ollu \'aviissaid. \\ --> Husk kongruens mellom subjekt og verbal.
\item Mun lohken ollu \'aviissat. \\ --> Objektet skal være i akkusativ.
\item Don lohket ollu \'aviissaid. \\ --> Er du sikker på at du svarer i riktig person?
\end{itemize}

\newslide
\scalebox{0.10}[0.10]{\includegraphics{img/skovi.png}} 
\newslide
Mii geavahit min máhtu:
\begin{itemize}
\item sámegiela syntávssa birra		
\item ohppiid gaskagiela birra
\end{itemize}

\newslide
\textbf{Sámegiela syntáksa}\\
omd. maid NP sáhttá sisttisdoallat:
\begin{itemize}
\item \small{NP: Pron A N Num Adv A CC Adv A N}  \\ 	
\textit{Mu boares áhku guokte hui stuora ja hirbmat váralaš beatnaga}	
\item Makkár kongruensa das galgá leat.
\end{itemize}



\newslide
\textbf{Lunddolaš ságastallan:} \\

\scalebox{0.40}[0.40]{\includegraphics{img/lgiella1.png}} 

\newslide
\textbf{Lunddolaš ságastallan:} \\

\scalebox{0.40}[0.40]{\includegraphics{img/lgiella2.png}} 

\newslide
\textbf{Lunddolaš ságastallan:} \\

\scalebox{0.40}[0.40]{\includegraphics{img/lgiella3.png}} 
\newslide
\textbf{Lunddolaš ságastallan:} \\

\scalebox{0.40}[0.40]{\includegraphics{img/lgiella4.png}} 
\newslide
\textbf{Lunddolaš ságastallan:} \\

\scalebox{0.40}[0.40]{\includegraphics{img/lgiella5.png}} 
\newslide
\textbf{Lunddolaš ságastallan:} \\

\scalebox{0.40}[0.40]{\includegraphics{img/lgiella6.png}} 


\newslide
\textbf{Čuolmmat -- 1: Didaktihkka versus pragmatihkka} \\
Mii háliidit geavaheaddji hárjehallat buot persovnnaid ja loguid. Dan dihte:
\begin{itemize}
\item{Ellipsa ii leat dohkálaš}
\item{Finihtta vearba lea bákkolaš}
\item{Ferte vástidit seammá vearbbain dalle go lea lunddolaš dan dahkat}
\item{Ii leat fátmmasteaddji 1. p duála ja plurála }
\item{\textit{In dieđe} ii leat dohkálaš vástádus}
\end{itemize}

\newslide
\textbf{Čuolmmat -- 2: Cealkagis ii leat finihtta vearba} \\

\textit{*Mun vuolggan ihttin.}\\
--> Svaret ditt må alltid inneholde et finitt verb. 

\newslide
\textbf{Vejolaš čoavddus:}\\

\textit{*Mun vuolggan ihttin.}\\
--> Svaret ditt må alltid inneholde et finitt verb. Kan det være en skrivefeil?


\newslide
\textbf{Čuolmmat -- 3: Cealkagis leat guokte finihtta vearbba:} \\

\textit{*Mun áiggun vuolggán.}\\
\textit{Mun boran haman.}\\
-- finihtta-finihtta-konstrukšuvnnas vearbbain galgá leat seammá sojaheapmi\\
-- ii galgga leat advearba gaskkas

\newslide
\textbf{Vejolaš čoavddus:} \\
Semánttalaš seahtta:\\
LIST INFV =  astat ádjánit áigut álgit beassat berret bivvat .... \\
\textnormal{*(INFV finihtta) + (VERB finihtta)}


\newslide
\textbf{Čuolmmat -- 4: Nominatiiva versus akkusatiiva} \\
Eat sáhte luohttit sátneortnegii, ja subjeakta ii leat bákkolaš
\begin{itemize}
\item Jearaldat jearrá objeavtta (muhto lea muhtumin vejolaš vástidit objeavtta haga)
\end{itemize}
\newslide
\textbf{Vejolaš čoavddus:} \\
Defineret vearbbaid ja bidjat daid semánttalaš seahtaide, omd: 
\begin{itemize}
\item vearbbat main lea objeakta bákkolaš argumeantan  (Strict Transitive Verbs)
\item vearbbat main ii sáhte leat HUMAN objeaktan 
\end{itemize}

\newslide
\textbf{Sáhttá go HUMAN leat objeaktan?} \\

 \textit{borrat}   - HUMAN sáhttá leat subjeakta, iige objeakta \\
 
  \textit{lohkat}  - seammá, muhto objeakta sáhttá leat namma, \\ omd   \textit{Ikte mun lohken Fosse.}

\newslide
\textbf{Čuolmmat -- 5: Čállinmeattáhusat} \\
\begin{enumerate}
\item sátnehápmi ii gávdno: \\ --> X finnes ikke i vårt leksikon. Kan det være en skrivefeil?
\item áiggukeahtes lemma
\item áiggukeahtes sojaheapmi
\end{enumerate}

\newslide
\textbf{Lasihit kásusgeahčosa njuolgga vuođđohápmái:} \\
Geahččalin 1500 dábálaš substantiivvain 
\begin{itemize}
\item (lokatiiva) -s/-is njuolgga Nom-hápmái: \\
57 \% áiggukeahtes sojaheapmi (PxSg3 - omd \textit{viessus})  \\  0,5 \% áiggukeahtes lemma \\  (omd \textit{eanas (eatnamis)} Adv dahje \\ vearba \textit{-stit} -- imperatiiva, vearbagenetiiva, biehttalanhápmi omd \textit{čogus (čohkumis)} )
\item (illatiiva) -i njuolgga Nom-hápmái: \\
0 \% áiggukeahtes sojaheapmi \\  2,3 \% áiggukeahtes lemma \\ (eanaš vearba -- preterihtta Sg3, omd \textit{báddii (báddái)}) 
\end{itemize}



\newslide
\textbf{Čuolmmat -- 5a: Čállinmeattáhus dahká áiggukeahtes lemma} \\
omd\textit{viessut}: \textit{viessut} Inf dahje \textit{viessat} Imprt \\
muhto mii árvidit ahte geavaheaddji háliida čállit \textit{viesut} N Pl Nom. \\
Vejolaš čovdosat:

\begin{itemize}
\item{Váldit eret problemáhtalaš lemmaid dahje sátnehámiid}
\item{Jearrat geavaheaddjis: \\ Goappá don oaivvildat? \textit{viesut}  N dahje \textit{viessut} V }?
\end{itemize}


\newslide
\textbf{Čuolmmat -- 5b: Čállinmeattáhus dahká áiggukeahtes sojaheami} \\
omd oamastangehčosat \\
\textit{biilas N Sg Nom Px Sg3 / biillas N Sg Loc} \\

Vejolaš čovdosat:
\begin{itemize}
\item Váldit eret oamastangehčosiid.
\item Kommenteret geavaheaddjái: \\
--> Mener du lokativ? I så fall er det feil stadieveksling.
\end{itemize}


%\newslide

%QA information retrieval how to process questions
%interactive, aske the user for clarification

%QA: 

%Question matrixes: generate a set of questions

%\newslide
%\textbf{Ovdamearka geavaheaddjis}
%The output is manipulated  - it would not give two mappings to the same reading.
%\scalebox{.32}[.35]{\includegraphics{img/sentence_example.png}}

\newslide
\textbf{Buoret ahte soames áššit báhcet divukeahttá go divvut dakkár mii leat riekta} \\
- muhto makkár vuorddámušat leat geavaheaddjis?

\newslide
\textbf{Evalueren ja buorideapmi}
\begin{itemize}
\item{Responsa geavaheddjiin}
\item{Responsa oahpaheddjiin}
\item{Vasta-log interneahtas}
\end{itemize}


\end{slide}
\end{document}




