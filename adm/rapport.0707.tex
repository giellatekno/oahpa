\documentclass[a4paper,norsk]{article}
\usepackage{babel}
\usepackage{ucs}
\usepackage[utf8x]{inputenc}
\usepackage[T1]{fontenc}
\usepackage{a4wide}

\usepackage{hyperref}
\usepackage{graphics}
\usepackage{url}

\begin{document}


\title{Interaktiv samisk arena på internett: Rapport, juli 2007}


\author{Trond Trosterud ja Lene Antonsen\\
Humanisttalaš fakultehta\\
Romssa Universitehta}

\maketitle


Prosjektet \textit{Interaktiivalaš sámegiel arena interneahtas - hárjehallanbáiki álggahalliidde ja daidda geat juo máhttet olu} har to mål:

\begin{enumerate}
\item Lage interaktive pedagogiske spel for samisk
\item Utvide repertoaret av samiske setningar på den interaktive spelplattforma \url{http://visl.sdu.dk}
\end{enumerate}

\section{Arbeid til no}

\subsection{Interaktive pedagogiske spel}

Etter ein del initial planlegging våren 2007, kom prosjektet i gang for fullt først 1.6.2007 da prosjektets pedagog hadde mulighet til å begynne for fullt. Så langt har programmerar, samiskpedagog og datalingvist arbeidd fram to dokument: Eit dokument som inneheld eit inventar over kva som trengst for dialogsystemet, og eit designdokument for korleis dette skal implementerast.   Vi har vurdert tilsvarande system for andre språk, m.a. maori, og sett opp prinsipp for dialaogstyring.

\subsection{Samiske setningar for visl}

Her har vi innleidd diskusjonar med forfattarar av samiske pedagogiske læremiddel, der målet er å inkludere deira setningar, slik at elevane deira kan bruke arbeidsoppgåvene frå læreboka ikkje berre som tradisjonell lekse, men også interaktivt på nettet.

Vi har studert korleis andre språkversjonar har lagt opp sine setningar (jf. t.d. \url{http://www.tekstlab.uio.no/grei/}, og vi har starta med å skrive eigne setningar.

\section{Arbeid i 2007}

Hausten 2007 vil det berre vere samiskpedagog og datalingvist i arbeid med prosjektet. Hovudvekta av arbeidet vil vere på visl-delen, der vi stør oss på programmerarar frå Syddansk Universitet.

Når det gjeld dialogsystemet vil vi arbeide med semantiske sett, og med dialogrammer. I dag generer programmet vårt alle korrekte former, til det pedagogiske programmet treng vi ein generator som alltid generer berre ei form. Vi vil derfor lage tre ulike versjonar av generatoren, ein for kvar hovuddialekt. Slik vil brukarane kunne velge på kva for ein hovuddialekt dei ynskjer at programmet skal kommunisere med dei på.

\section{Resten av prosjektperioden}

Prosjektet er planlagt å gå ut 2008. Hovudvekta i 2008 vil vere på dialogsystemet, visl-plattforma reknar vi med å ha gjort hovudtyngda av arbeide på i 2007. 


\end{document}